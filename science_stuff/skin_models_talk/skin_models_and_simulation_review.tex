%% Presentation about models and simulations of the human skin
%% reviewing a paper by Mitragotri et al.
%%
%% presentation by Torbjörn Klatt < science at torbjoern minus klatt dot de >
%% 2012-07-10
%% Creative Commons 3.0 Attribution-ShareAlike

\documentclass[utf8x,compress,professionalfonts]{beamer}

\usepackage{pgfpages}
\usepackage{eurosym}
\usepackage{amsmath}
\usepackage{wrapfig}

\usetheme[width=4em]{Berkeley}
\usecolortheme{dove}
\setbeameroption{hide notes}
\usenavigationsymbolstemplate{}

\renewcommand{\vec}[1]{{\mathbf #1}}
\newcommand{\eref}[1]{(\ref{#1})}

\title[Simulation der Diffusion durch die Haut]{Modelle zur Simulierung der Diffusion durch die Haut}
\author{Torbj\"orn Klatt}
\institute{Goethe-Universi\"at Frankfurt am Main}
\logo{\includegraphics[width=1.25cm]{goethe_uni.png}}
\date{10.07.2012}
\usefoottemplate{\vbox{%
  \tinycolouredline{structure!20}%
    {\color{gray}\tiny\textrm{\insertdate\hfill Creative Commons 3.0 BY-SA}}%
}}

\begin{document}

\begin{frame}
  \begin{center}
    {\Large \textbf{Modelle zur Simulierung der Diffusion durch die Haut} \par}
    \bigskip

    {\scriptsize Review eines Papers von \textsc{Mitragotri} et al. \par}
    \bigskip

    {\small \textit{von Torbj\"orn Klatt} \par}
    \bigskip

    {\scriptsize Frankfurt a.M., 10. Juli 2012 \par}
    \bigskip

    {\scriptsize Im Seminar: \textit{Transport und Diffusion im Biogewebe} \par}
    \bigskip

    \includegraphics[width=0.25\textwidth]{goethe_uni.png}
  \end{center}
\end{frame}

\section{Motivation}
\begin{frame}{Am Anfang stehen Fragen \dots}
  \begin{itemize}
    \item<1-> Wie funktionieren Salben?
      \bigskip
    \item<2-> Wie genau ist das Aufquellen der Fingerkuppen unter Wasser zu
      erkl\"aren?
      \bigskip
    \item<3-> Wie sch\"utzt die Haut den K\"orper vor Schadstoffen?
      \bigskip
    \item<4-> Wie lassen sich Medikamente schonend transdermal verabreichern?
      \bigskip
    \item<5-> \dots
  \end{itemize}
\end{frame}

\begin{frame}{\"Uber die Haut}
  \begin{itemize}
    \item mehr-schichter Aufbau
      \begin{itemize}
        \item \textit{Epidermis}, Oberhaut
          \begin{itemize}
            \item \textbf{Stratum Corneum, Hornschicht} (SC)
            \item \dots
          \end{itemize}
        \item \textit{Dermis}, Lederhaut
        \item \textit{Subcutis}, Unterhaut
      \end{itemize}
      \bigskip
    \item nicht gleichm\"a\ss{}ig, da mit Poren und Haarwurzeln \"uberzogen
  \end{itemize}
\end{frame}

\section{Modell\"ubersicht}
\subsection{Einfache Modelle}
\begin{frame}{Die Grundlagen}
  \begin{itemize}
    \item<1-> Wissen und Ausnutzung der Hauteigenschaften bereits in der fr\"uhen
      Antike
      \bigskip
    \item<2-> gr\"o\ss{}te Entwicklungen und Fortschritte in den letzten 70 Jahren
      \bigskip
      \begin{itemize}
        \item<3-> in den 60ern formulierte \textsc{Higuchi} erste mathematische
          Modelle auf Basis von \textsc{Fick}s erstem Gesetz
          \begin{equation}
            J=-D\frac{\partial c}{\partial x}
          \end{equation}
      \end{itemize}
  \end{itemize}
\end{frame}

\subsubsection{Station\"ar}
\begin{frame}{Einfachste Modelle}
  Am Anfang steht immer ein einfaches Modell, so auch hier.

  \vfill

  \textbf{Annahmen:}
  \begin{itemize}
    \item SC ist homogen (mikro- und makroskopisch)
      \bigskip
    \item SC ist \"uberall gleich dick
      \bigskip
    \item Eigenschaften der SC sind zeit- und ortsunabh\"angig
  \end{itemize}
\end{frame}


\begin{frame}{Fluss gel\"oster Teilchen durch die Haut}
  Unter Annahme, dass die Haut eine quasi-homogene Membran ist, l\"asst sich
  der Fluss gel\"oster Teilchen durch die Haut im Flussgleichgewicht schreiben als
  \begin{equation}
    J_{ss}=\frac{\frac{C_s}{C_v}D\Delta C_v}{h}
  \end{equation}
  mit Konzentration der L\"osung ($C_s$), Konzentration in den Hautzellen ($C_v$),
  dem Diffusionskoeffizenten des \textit{SC} mit Dicke $h$ und dem Konzentrationsgradienten
  ($\Delta C_v$) durch die Haut.
\end{frame}

\begin{frame}{Wie durchl\"assig ist denn die Haut?}
  Das \textit{Stratum Corneum} ist die am wenigsten durchl\"assige Schicht und der
  limitierende Faktor.
  \bigskip

  Die Flussgeschwindigkeit $J_{ss}$ h\"angt von der Art der gel\"osten Teilchen
  sowie dem Konzentrationsverhältnis zwischen L\"osung und Hautzellen ($K=\frac{C_s}{C_v}$)
  ab.
  \bigskip

  Daraus l\"asst sich ein Durchl\"assigkeitskoeffizient berechnen:
  \begin{equation}
    \kappa_p = \frac{KD}{h}
  \end{equation}
\end{frame}

\begin{frame}{Bessere Modelle f\"ur $\kappa_p$ (QSPR)}
  \begin{itemize}
    \item<1-> In den 90er entwickelten haupts\"achlich \textsc{Potts} und \textsc{Guy}
      ausgefeiltere Modelle f\"ur $\kappa_p$ und $J_{max}$
      $$log(\kappa_p)=log\left(\frac{D_0}{h}\right)+log(K)-\left[\frac{\beta V}{2.303}\right]$$
    \item<2-> Die sogenannten Modelle zur quantitativen Bestimmung des
      \textit{Struktur-Durchl\"assigkeits-Verh\"altnisses} (\textit{QSPR}) bilden
      experimentelle Werte sehr gut ab.
      \bigskip
    \item<3-> \textbf{Nachteil:} Funktionieren nur unter bestimmten vereinfachenden
      Annahmen und ohne biochemischer Interaktion der Teilchen mit dem SC.
  \end{itemize}
\end{frame}


\begin{frame}{\textit{Ziegel-M\"ortel-Modell}}
  \begin{itemize}
    \item<1-> SC besteht aus einzelnen Corneozyten, die \"uber Lipiddoppelschichten
      untereinander verbunden sind.
      \bigskip
    \item<2-> \textbf{Annahme:} Diffusion \textbf{nur} durch Lipidschicht, nicht
      durch Zellen
      \bigskip
    \item<2-> Dann wird die Durchl\"assigkeit definiert durch:
      \begin{equation}
        \kappa_p = \frac{\alpha D_{lip}K_{lip}}{h_{lip}}
      \end{equation}
      mit $h_{lip}$ als effektiver L\"ange des Diffusionswegs.
  \end{itemize}
\end{frame}

\begin{frame}{Erweiterungen des \textit{Ziegel-M\"ortel-Modells}}
  \begin{itemize}
    \item<1-> \textbf{Nachteil des vorherigen Modells}:\\
      Anteil der Diffusion \textit{durch} die Corneozyten wurde ignoriert.
      \bigskip
    \item<2-> Hinzunahme der anteiligen Diffusion erh\"oht Komplexit\"at des Modells
      gewaltig.
      \bigskip
    \item<3-> Als Kompromiss werden die Diffusions- und Durchl\"assigkeitskoeffizienten
      sowie die L\"ange des Diffusionswegs \"uber das gesamte SC gemittelt.
  \end{itemize}
\end{frame}

\begin{frame}{Por\"ose Diffusionswege f\"ur hydrophile Teilchen}
  \begin{itemize}
    \item<1-> Vorherige Modelle ignorieren den Effekt von Poren und Haarfolikel, die
      einen nennenswerten Anteil der Hautoberfl\"ache einnehmen.
      \bigskip
    \item<2-> Koeffizienten f\"ur \textit{Por\"osit\"at}, \textit{Tortousit\"at} und
      Durchmesser der Poren/Folikel sowie der diffundierenden Teilchen k\"onnen
      hinzugenommen werden.
      \bigskip
    \item<3-> \textbf{Nachteil:} Funktioniert fast ausschlie\ss{}lich nur f\"ur stark
      hydrophile Teilchen.
  \end{itemize}
\end{frame}


\subsubsection{Zeitabh\"angig}
\begin{frame}{Pharmakokinetische Modelle (1)}
  \begin{itemize}
    \item \textbf{Annahme:} K\"orper und Haut sind voneinander getrennte Reservoirs
      f\"ur Wirkstoffe
      \bigskip
    \item \textbf{Vorteil:} Gew\"ohnliche DGLen beschreiben Flussraten zwischen
      verschiedenen Kompartments
  \end{itemize}
  \scriptsize
  \begin{align}
    V_{SC}\frac{d\left[ C_{skin}\right] }{dt} &= k_1C_v - k_{-1}\left[ C_{SC}\right]-k_2\left[ C_{SC}\right]+k_{-2}\left[ C_{VE}\right]\\
    V_{VE}\frac{d\left[ C_{VE}\right] }{dt}   &= k_2\left[ C_{SC}\right] -k_{-2}\left[ C_{VE}\right] -k_3\left[ C_{VE}\right] +k_{-3}C_b
  \end{align}
  \normalsize
\end{frame}

\begin{frame}{Pharmakokinetische Modelle (2)}
  \begin{itemize}
    \item Experimentelle Daten werden ohne Ber\"ucksichtigung wichtiger Hautparameter
      angeglichen: $\kappa_p$, $D$, $K$, $h$
      \bigskip
    \item Au\ss{}erdem werden nicht alle Aspekte des \textsc{Fick}'schen ersten Gesetz
      abgebildet.
  \end{itemize}
\end{frame}

\begin{frame}{Wirkstoffe interagieren mit SC}
  \begin{itemize}
    \item<1-> Bisherige Modelle ignorieren bekannten Effekt der bio-chemischen Bindung
      von diffundierenden Teilchen in den SC-Zellen.
      \bigskip
    \item<2-> Bindung verlangsamt Diffusion und l\"asst sich als System gekoppelter PDEs
      beschreiben:
      \begin{align}
        \frac{\partial C_u}{\partial t} &= D\frac{\partial^2 C_u}{\partial x^2}-k_{on}C_u+k_{off}C_b \\
        \frac{\partial C_b}{\partial t} &= k_{on}C_u - k_{off}C_b
      \end{align}
  \end{itemize}
\end{frame}

\subsection{Komplexere}
\begin{frame}{N\"aher an der Realit\"at}
  \begin{itemize}
    \item Ausgefeiltere Modelle basierend auf den genannten und beschreiben
      zus\"atzlich:
      \bigskip
      \begin{itemize}
        \item<2-> Transport und Verstoffwechselung von Wirkstoffen durch/in SC
          \bigskip
        \item<3-> Gekoppelter Transport mehrerer Stoffe, die einander verst\"arken/hindern
          \bigskip
        \item<4-> zus\"atzliche kapillare Kr\"afte in SC
          \bigskip
      \end{itemize}
  \end{itemize}
\end{frame}

\section{Parameter}
\begin{frame}{}
  \begin{center}
    Gr\"o\ss{}te Herausforderung in L\"osung vorheriger Modelle ist Ermittlung
    der Hau(p)tparameter:
    \bigskip
    $$K\qquad D\qquad h$$
  \end{center}
\end{frame}

\subsection{Verteilungskoeffizient}
\begin{frame}{Ermittlung des Verteilungskoeffizienten $K$}
  \begin{itemize}
    \item Diffusion durch SC beinhaltet auch Wechsel zwischen verschiedenen Kompartments
      (Lipidschicht, Corneozyt, umgebendes Medium etc.)
      \bigskip
    \item F\"ur SC zu umgebendem Medium gilt:
      $$K_{SC/V} = \Phi_{lip}K_{lip/V} + \Phi_{cor}K_{cor/V}$$
      mit $\Phi$ als Volumenanteile von Lipiden und Corneozyten.
      \bigskip
    \item Letzteres l\"asst sich weiter in Anteile aus Wasser und Proteinen
      aufspalten.
  \end{itemize}
\end{frame}

\subsection{Diffusionskoeffizient}
\begin{frame}{Ermittlung des Diffusionskoeffizienten $D$ (1)}
  \begin{itemize}
    \item I.d.R. abh\"angig von Ort, Zeit, Konzentrationen und Diffusionsrichtung
      \bigskip
    \item \"Ublicherweise wird $D$ aus experimentellen Daten gewonnen
      \bigskip
    \item<2-> \textbf{Nachteil:}
      \begin{itemize}
        \item Experimentelle Daten sind oft sehr ungenau
        \item Derartige $D$ k\"onnen nur mit den gleichen Stoffen verwendet werden,
          f\"ur die sie ermittelt wurden.
      \end{itemize}
      \bigskip
    \item<3-> \textbf{Alternative:} Modellbasierte Errechnung von $D$
  \end{itemize}
\end{frame}

\begin{frame}{Ermittlung des Diffusionskoeffizienten $D$ (2)}
  \begin{itemize}
    \item<1-> \textsc{Potts}-\textsc{Guy}-Modell:
      \scriptsize
      \begin{equation}
        \frac{D_{SC}}{h_{SC}} = \left( \frac{D^0}{h_{SC}}\right) e^{-\beta''\cdot MW}
      \end{equation}
      \normalsize
    \item<2-> \textsc{Wang}-\textsc{Kasting}-\textsc{Nitsche}-Modell:
      Ber\"ucksichtigung mikroskopischer Partitionierung und durchqueren dieser
      \bigskip
    \item<3-> \textsc{Mitragotri}-Modell: Basierend auf mechanischer Statistik und
      ben\"otigter Energie eines diffundierenden Teilchens
  \end{itemize}
\end{frame}


\subsection{Wegl\"ange und -kr\"ummung}
\begin{frame}{Ermittlung der Diffusionswegl\"ange}
  \begin{itemize}
    \item<1-> Wegl\"ange ist Abh\"angig von Annahme, ob Corneozyten durchl\"assig sind
      oder nicht.
      \bigskip
    \item<2-> Bei \textbf{durchl\"assigen} Corneozyten ist der Weg \textbf{k\"urzer}
      und die Kr\"ummung \textbf{zu vernachl\"assigen}.
      \bigskip
    \item<3-> Bei \textbf{un}durchl\"assigen Corneozyten ist Weg \textbf{dramatisch l\"anger}
      und Tortousit\"at nimmt \textbf{entscheidenden Einfluss}.
  \end{itemize}
\end{frame}


\section{L\"osungsans\"atze}
\begin{frame}{Wie die Berechnungen durchf\"uhren?}
  \begin{itemize}
    \item<1-> Mit welchen mathematischen Methoden k\"onnen die beschriebenen Modelle
      berechnet werden?
      \bigskip
    \item<2-> Welche Vor- bzw. Nachteile ergeben sich aus der mathematischen Betrachtung
      der Modelle?
      \bigskip
  \end{itemize}
\end{frame}

\subsection{Analytisch}
\begin{frame}{\textsc{Laplace}-Transformation}
  \begin{itemize}
    \item Einfache Modelle k\"onnen oft mit \textsc{Laplace}-Transformationen
      ann\"ahernd analytisch gel\"ost werden.
      \bigskip
    \item Wird meist zur Berechnung der Konzentrationen in Abh\"angigkeit von Ort
      und Zeit verwendet.
      \bigskip
    \item<2-> \textbf{Nachteil:} K\"onnen nur bei konzentrationsunabh\"angigen Koeffizienten
      und zeitunabh\"angigen Diffusionskoeffizienten verwendet werden.
  \end{itemize}
\end{frame}

\subsection{Numerisch}
\begin{frame}{Finite Differenzen}
  \begin{itemize}
    \item Was FD sind, ben\"otige ich in diesem Haus hier nicht zu erkl\"aren.
      \bigskip
    \item Angewendet auf zeitabh\"angiges Modell, bei dem der Diffusionskoeffizient
      $D$ von der Konzentration $C$ und diese wiederum vom Ort abh\"angen.
      \bigskip
    \item<2-> \textsc{Kasting} verwendete ein gestaffeltes Gitter zur Diskretisierung.
  \end{itemize}
\end{frame}

\begin{frame}{Finite Elemente/Volumen}
  \begin{itemize}
    \item<1-> 2-Komponenten-Modell mit Fluss durch die Grenzfl\"ache und
      unterschiedlichen Diffusionskoeffizienten (\textsc{Rim})
      \bigskip
    \item<2-> 2-Phasen Ziegel-M\"ortel-Modell und Wirkstofffluss durch SC (\textsc{Heisig})
      \bigskip
    \item<3-> FE Simulationen legen nahe, ein einfaches Ziegel-M\"ortel-Modell an
      Stelle komplexer Geometrien zu verwenden. (\textsc{Barbero})
  \end{itemize}
\end{frame}


\section{Ausblick}
\begin{frame}{Derzeitige Anwendung und Planungen}
  \begin{itemize}
    \item Heutzutage werden haupts\"achlich die QSPR-Modelle verwendet.
      \bigskip
    \item Hauptanwendung ist Berechnung von Aufnahme von Schadstoffen \"uber die
      Haut bei unterschiedlich langer Exposition.
      \bigskip
    \item Zahlreiche Grenzwerte von Gesundheitsbeh\"orden sind so berechnet worden
      (z.B. Zeitbegrenzung f\"urs Schwimmen in mit Pestiziden belastetem Wasser)
  \end{itemize}
\end{frame}

\begin{frame}{Was bleibt zu tun?}
  \begin{itemize}
    \item<1-> Wahl des Modells ist sehr stark abh\"angig von der Fragestellung.
      \bigskip
    \item<2-> Komplexe Modelle m\"ogen die Realit\"at besser widerspiegeln, doch
      fehlt hierf\"ur die experimentelle Datengrundlage.
      \bigskip
    \item<3-> Daher m\"ussen die Zahl und Art freier Modellparameter an diese
      Datenlage angepasst werden, um Validierung zu erm\"oglichen.
  \end{itemize}
\end{frame}


\section{Ende}
\begin{frame}{Danke}
  \begin{center}\Large
    Vielen Dank f\"ur die Aufmerksamkeit

    \vfill

    Fragen ? !
  \end{center}
\end{frame}


\section{Quellen}
\begin{frame}{Quellen}
  \begin{itemize}
    \item \textsc{Mitragotri et al.},
      \textit{Mathematical Models of Skin Permeability: An Overview},
      Int. J. Pharm., 418(2011)115-29, doi:10.1016/j.ijpharm.2011.02.023.
  \end{itemize}
\end{frame}


\end{document}
